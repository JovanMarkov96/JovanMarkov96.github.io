%%%%%%%%%%%%%%%%%%%%%%%%%%%%%%%%%%%%%%%%%

%%%%%%%%%%%%%%%%%%%%%%%%%%%%%%%%%%%%%%%%%
% ModernCV-based wrapper template for site CV (EXACTLY matching your original main.tex)
%%%%%%%%%%%%%%%%%%%%%%%%%%%%%%%%%%%%%%%%%
\documentclass[11pt,a4paper,sans]{moderncv}

\moderncvstyle{classic} % CV theme - options include: 'casual' (default), 'classic', 'oldstyle' and 'banking'
\moderncvcolor{blue} % CV color - options include: 'blue' (default), 'orange', 'green', 'red', 'purple', 'grey' and 'black'
\usepackage{multicol}
\usepackage{lipsum} % Used for inserting dummy 'Lorem ipsum' text into the template
\usepackage[hyperref]{}
\usepackage[scale=0.9]{geometry} % Reduce document margins
%\setlength{\hintscolumnwidth}{3cm} % Uncomment to change the width of the dates column
%\setlength{\makecvtitlenamewidth}{10cm} % For the 'classic' style, uncomment to adjust the width of the space allocated to your name
\usepackage{orcidlink}
\hypersetup{
    colorlinks=true,
    urlcolor=blue}
    
\usepackage{pstricks}
\usepackage{pst-barcode}

% Use local style files (assumes they are in ./moderncv_styles/)
% (LaTeX will find .sty files in the working directory)
% Pandoc compatibility for compact lists
\providecommand{\tightlist}{\setlength{\itemsep}{0pt}\setlength{\parskip}{0pt}}

% Name and contact
\firstname{Jovan}
\familyname{Markov \orcidlink{0000-0003-1611-7150}}
\email{jovan.markov@weizmann.ac.il}
\mobile{(+972) 058 790 7328}
\extrainfo{\emailsymbol\emaillink{jocin.meil@gmail.com}}

% Uncomment to add photo (ensure pictures/picture.jpg exists)
% \photo[70pt][0.4pt]{pictures/picture}

\begin{document}

\makecvtitle

\section{Education}

\cventry{2022 -- present}{Weizmann Institute of Science}{\newline Physics}{PhD, Quantum computing and quantum simulation with trapped ions}{\newline Advisor: Prof. Roee Ozeri}{} 

\cventry{2019 -- 2021}{Weizmann Institute of Science}{\newline Physics}{MSc}{\newline Advisor: Prof. Kfir Blum}{} 

\cventry{2015 -- 2019}{Faculty of Physics, University of Belgrade}{\newline Theoretical and Experimental Physics}{BSc}{}{GPA -- 9.79/10} 

\section{Skills}\label{skills}

\cvitem{}{Git}
\cvitem{}{Jupyter Notebook}
\cvitem{}{LaTeX}
\cvitem{}{Linux command line}
\cvitem{}{Mathematica}
\cvitem{}{Matlab}
\cvitem{}{Python}
\cvitem{}{ROOT}
\cvitem{}{Event Planning}
\cvitem{}{Outreach}
\cvitem{}{Public Speaking}
\cvitem{}{Presenting}
\cvitem{}{Project Management}
\cvitem{}{Seminar Organization}

\section{Publications}\label{publications}

\{\% for post in site.publications reversed \%\} \{\% include
archive-single-cv.html \%\} \{\% endfor \%\}

\section{Teaching, Organizing}\label{teaching-organizing}

\cventry{2024 -- 2025}{Organizer of The AMOS Seminar}{Faculty of Physics and Faculty of Chemistry, Weizmann Institute of Science}{A weekly seminar for journal clubs, invited speakers, and PhD defenses}{}{}
\cventry{2020 -- 2022}{Tutor}{Faculty of Physics, Weizmann Institute of Science}{Helping new MSc students with coursework}{}{}

\section{Languages}\label{languages}

\cventry{}{Serbian}{Native}{}{}{}
\cventry{}{English}{Fluent}{TOEFL iBT Score:110/120 (Nov 2018)}{}{}
\cventry{}{Hebrew}{Elementary}{}{}{}

\section{Conferences, Seminars, Talks,
Posters}\label{conferences-seminars-talks-posters}

\cventry{2025}{ECTI 2025, European Conference on Trapped Ions}{Poster}{Digital predistortion of optical field of a fast and high-fidelity entangling gate for trapped ion qubits. Authors: J. Markov, Y. Shapira, N. Akerman, R. Ozeri. \href{https://www.ecti2025.com/home}{Conference website}}{}{}
\cventry{2025}{Petnica Science Center, Physics Seminar, Valjevo, Serbia}{Workshop}{Quantum Computing Workshop (two-day, 8-hour hands-on course based on \href{https://www.cl.cam.ac.uk/teaching/1819/QuantComp/materials.html}{Cambridge Quantum Computing syllabus}, plus trapped ion quantum computing and lab tour)}{}{}
\cventry{2023}{Annual New Year Seminar, Faculty of Physics, University of Belgrade, Serbia}{Talk}{Quantum computers and what to do with them. (in Serbian)}{}{}
\cventry{2023}{54th Annual Meeting of the APS Division of Atomic, Molecular and Optical Physics, Spokane, Washington}{Talk}{Digital predistortion of optical field of a fast and high-fidelity entangling gate for trapped ions qubits. Authors: \textbf{JM}, Yotam Shapira, Nitzan Akerman, Roee Ozeri.}{}{}
\cventry{2023}{25th Annual SQuInT Workshop, Albuquerque, New Mexico}{Poster}{Programmable quantum simulations on a trapped-ions quantum computer with a global drive. Authors: \textbf{JM}, Yotam Shapira, Nitzan Akerman, Ady Stern, Roee Ozeri.}{}{}
\cventry{2018}{IPPOG CMS Masterclass, Belgrade, Serbia}{Workshop}{Helping participants with hands-on data-analysis exercises.}{}{}

\section{Projects, Internships}\label{projects-internships}

\cventry{2021}{MSc thesis}{Project: Ultralight dark matter: Ground state solutions in the presence of two fields}{Advisor: Prof. Kfir Blum}{}{}
\cventry{2019}{Summer Student Program, CERN, Geneva, Switzerland}{Project: \href{https://cds.cern.ch/record/2687408/files/Jovan_Markov_CERN_Summer_School_2019_project_report%20final.pdf}{Top quarks in the lepton + jets channel in PbPb collisions}}{Mentors (CMS): Pedro Ferreira daSilva, Émilien Chapon}{}{}
\cventry{2017}{Summer Research Student, Weizmann Institute of Science, Rehovot, Israel}{Kupcinet-Getz International Summer School}{Mentor (ATLAS) Prof. Alexander Milov. Studied the production of Z bosons in the di-electron decay channel in proton-proton (pp) collisions which was measured in the ATLAS detector. The results from pp collisions were compared to the previously measured data in the proton-lead (p-Pb) collisions system at the same energy. Information on the nuclear modification factor $RpPb$, for Z bosons in p-Pb collisions was extracted from the comparison. The whole data analysis was done by writing C++ scripts using ROOT. \href{http://www.weizmann.ac.il/particle/atlas/sites/particle.atlas/files/uploads/heavy-ion-physics.pdf}{Booklet} with our projects.}{}{}
\cventry{2014 -- 2015}{High school Physics program, Petnica Science Center, Valjevo, Serbia}{Project: Determining the mass of the $K_S^0$, $\pi^+$, and $\pi^-$ mesons using the Theil index}{Mentors (LHCb): Vladimir Gligorov, Miriam Lucio Martinez, Diego Martinez Santos}{}{}
\cventry{2013}{High school Physics program, Petnica Science Center, Valjevo, Serbia}{Project: Examination of properties of hydraulic jumps}{Mentors: Vladan Pavlović, Marija Janković}{}{}

\section{Training/Courses}\label{trainingcourses}

\cvitem{2015}{IPPOG CMS Masterclass in Belgrade, Serbia}
\cvitem{2015}{Introduction to programming with MATLAB, Vanderbilt University (Coursera)}
\cvitem{2014}{LHCb Masterclass, Petnica Science Center, held by Vladimir Gligorov (LHCb collaboration)}

\section{Memberships}\label{memberships}

\cvitem{2015 -- present}{EMINETER - student organization of the Faculty of Physics, Belgrade University}
\cvitem{since 2015}{Mathematical Grammar School Alumni Society (ALMAGI)}
\cvitem{2017, 2018, 2024}{Serbian Physical Society}
\cvitem{2020 -- present}{Israel Physical Society}
\cvitem{2023}{American Physical Society}

\section{Awards}\label{awards}

\cvitem{2016}{1st place at student research paper competition in physics, at Primatiyada 2016, regional event for students from Natural Science and Mathematics faculties, held in Albena, Bulgaria.}
\cvitem{2015}{"Vuk Karadi" diploma, Ministry of Education of the Republic of Serbia.}
\cvitem{2012 -- 2015}{Won 3rd prize four years in a row at the national level highschool physics competition, organized by the Serbian Physical Society.}

\section{Scholarships}\label{scholarships}

\cvitem{2015 - 2019}{Scholarship for exceptionally gifted high school and university students, awarded by the Ministry of Education, Science and Technological Development of the Republic of Serbia.}
\cvitem{2013 - 2016}{Scholarship given to high school students for their achievements on the national and international competitions during the previous calendar year, awarded by the Fund for Young Talents of the Republic of Serbia.}
\cvitem{2013}{Scholarship given to high school students for their achievements on the renowned competitions at home and abroad during the year 2012, awarded by The City Of Belgrade - City Administration.}

\section{Other engagements}\label{other-engagements}

\cvitem{2016 - present}{Assistant at the Physics program, \href{http://petnica.rs/}{Petnica Science Center}, Serbia. Giving lectures and helping high school students working on research projects.}
\cvitem{2024}{Reviewer of 4th Grade problem sets, High School Physics Competitions (academic year 2023/24), organized by the Serbian Physical Society.}
\cvitem{2017 -- 2018}{Member of committee at the National Physics Competition, organized by the Serbian Physical Society.}
\cvitem{2015 -- 2019}{Engaged in science promotion activities with Faculty of Physics, University of Belgrade, demonstrating physics experiments at schools all across Serbia.}
\cvitem{2014 -- 2019}{Journalist and speaker for "Step towards science" radio show, at the national radio station Radio Belgrade 1, with my own segment "And that's physics", where I talk about a certain topic from physics.}
\cvitem{2013 -- 2019}{Science Demonstrator, Center for the Promotion of Science, Belgrade, Serbia.}
\cvitem{2012 -- 2017}{Researchers Night (part of European Researchers Night project), Belgrade, Serbia. Organized several exhibits over the years. Also participated in demonstrating popular science experiments since 2012.}
\cvitem{2012 -- 2018}{Serbian Science Festival, Belgrade, Serbia. Organized for several years the "Experteenager zone", part of the Festival where high schools participate. Organized several exhibits. Also participated in demonstrating popular science experiments since 2012.}
\cvitem{2012 -- 2017}{Initiated and organized a School Science Festival, at my former Elementary school, which has been held for five consecutive years and has been seen by several thousand visitors. It has more than 80 exhibits which I made myself.}
\cvitem{2010 -- 2017}{Writer at Viva fizika (Viva physics), a science popularization web portal in physics and astronomy.}

\section{Other Interests}\label{other-interests}

\cvitem{}{Hiking}
\cvitem{}{Photography}
\cvitem{}{Digital art}
\cvitem{}{History}
\cvitem{}{Language geek}

\section{References}\label{references}

\cvitem{}{Kfir Blum, Professor, Weizmann Institute of Science, Israel}
\cvitem{}{Alexander Milov, Professor, Weizmann Institute of Science, Israel}
\cvitem{}{Vladimir Gligorov, research scientist (CRCN grade), CNRS, France}

\section{Talks}\label{talks}

\{\% for post in site.talks reversed \%\} \{\% include
archive-single-talk-cv.html \%\} \{\% endfor \%\}



% Add QR code at the end using PNG image (pdflatex compatible)
% \section{CV QR code}
% \cvitem{}{\includegraphics[width=0.25\textwidth]{pictures/qrcv.png}}

\end{document}
